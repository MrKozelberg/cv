\documentclass[11pt,a4paper, russian]{moderncv}

\usepackage[utf8]{inputenc}
\usepackage[T2A]{fontenc}
\usepackage[english, main=russian]{babel}
\usepackage{url}

\renewcommand{\familydefault}{\sfdefault}

\moderncvtheme[grey]{classic}

\usepackage[scale=0.9]{geometry}
\recomputelengths

\firstname{Александр}
\familyname{Козлов}

\mobile{+7 999 139 18 29}
\email{kozlov.alexander.vladimirovich@gmail.com}
\extrainfo{\url{https://github.com/MrKozelberg}}

\photo[64pt]{photo}

\begin{document}

\maketitle

\section{Образование}
\cventry{2018--2022}{Бакалавриат}{Высшая школа общей и прикладной физики}{Нижегородский государственный университет им. Н. И. Лобачевского}{}{Направление: Физика (03.03.02)}

\section{Опыт работы}
\cventry{2020--н.в.}{Лаборант-исследователь}{Институт прикладной физики Российской академии наук}{Нижний Новгород}{}{Лаборатория электромагнитного окружения Земли (Лаб. 265)}

\section{Сфера деятельности}
\cventry{2020--2021}{Создание столбчатой модели глобальной атмосферной электрической цепи}{}{\emph{Важные результаты:} созданная модель позволяет учесть влияние крупномасштабных возмущений проводимости воздуха на распределение электрических полей в атмосфере}{}{}

\cventry{2021--2022}{Исследование влияния колебания Маддена–Джулиана на глобальную атмосферную электрическую цепь}{}{\emph{Важные результаты:} обнаружена ранее не известная связь атмосферного электричества с колебанием Маддена–Джулиана; установлено, что как моделируемый ионосферный потенциал, так и измеряемый на станции Восток градиент потенциала имеют устойчивые синусоидальные вариации по фазам колебания Маддена–Джулиана; с помощью метода эмпирических ортогональных функций найден физический механизм, отвечающий за наблюдаемую синусоидальную вариацию электрического поля на масштабах колебания Маддена–Джулиана}{}{}

\section{Публикации и выступление на конференциях}
\cvlistitem{Участие в конференции 17th International Conference on Atmospheric Electricity с постерным докладом Patterns Related to The Madden–Julian Oscillation in The Global Electric Circuit Variation (см. \href{https://www.researchgate.net/publication/363158363_Patterns_related_to_the_Madden-Julian_Oscillation_in_the_global_electric_circuit_variation}{\emph{постер}} и \href{https://www.icae2022.com/program}{\emph{программу конференции}})}
\cvlistitem{Проходит процедуру рецензирования в журнале Atmospheric Research статья: Alexander V. Kozlov, Nikolay N. Slyunyaev, Nikolay V. Ilin, Fedor G. Sarafanov, Alexander V. Frank-Kamenetsky, \textbf{The effect of the Madden–Julian Oscillation on the global electric circuit}}


\section{Область научных интересов}
Атмосферное электричество, матричные методы в климатологии и метеорологии, климатические моды

\section{Технические навыки}
\cvlistitem{Статистический анализ временных рядов, знание метода эмпирических ортогональных функций, wavelet-анализа и ряда других методов анализа данных}
\cvlistitem{Написание скриптов на языке Python (NumPy, Matplotlib, Pandas, TensorFlow), работа в Jupyter-notebook}
\cvlistitem{Работа в Linux}
\cvlistitem{Оформление научных работ, презентаций и постеров в LaTeX}
\cvlistitem{Опыт работы на языке C++: создание столбчатой модели глобальной электрической цепи (\url{https://github.com/MrKozelberg/gec})}

\section{Прочее}
\cvlistitem{Диплом призёра международной олимпиады Petropolitan Science (Re)Search по направлению Физика (Рег. No 03-22-0017)}
%\cvlistitem{Грамота за достигнутые успехи в военной подготовке, за высокую дисциплину и в честь 74-ой годовщины Победы в Великой Отечественной войне (получена в ходе обучения в Военном учебном центре ННГУ им. Н. И. Лобачевского)}

\end{document}
